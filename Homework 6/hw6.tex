%%%%%%%%%% DOCUMENT STUFF %%%%%%%%%%

\documentclass[10pt,letterpaper]{article}
\usepackage{mathtools}
\usepackage{amsmath}
\usepackage{amssymb}
\usepackage{datetime}
\usepackage{setspace}
\usepackage{tkz-graph}
\usepackage[margin=1in]{geometry}

%%%%%%%%%% FORMATTING %%%%%%%%%%

\newdate{date}{20}{03}{2017}
\date{\displaydate{date}}
\doublespacing
\setcounter{secnumdepth}{0}
\newcommand\tab[1][0.5cm]{\hspace*{#1}}
\newcommand*\circled[1]{\tikz[baseline=(char.base)]{
            \node[shape=circle,draw,inner sep=2pt] (char) {#1};}}

%%%%%%%%%% CONTENT %%%%%%%%%%

%%%%% COVER PAGE %%%%%

\begin{document}
\title{CS 180: Homework 6}
\author{
	Jonathan Woong\\
	804205763\\
	Winter 2017\\
	Discussion 1B}
\maketitle
\pagebreak

%%%%% PROBLEM 1 %%%%%

\section{Problem 1}
Suppose that you are given as input a list of $n$ birthdays (in the format $MMDDYYYY$) give an algorithm to check if two people in the list have the same birthday. Your algorithm should always be correct and run in expected time $O(n)$.\\\\
Let $U$ be the set of all possible birthday combinations $b$.\\
Let $h(b)$ be the hasing function that maps a birthday $b$ to an integer in $\{1,\dots,n\}$.\\
Let $\mathcal{D}$ be the database that stores hashed birthdays obtained by $h(b)$.\\
Let $L$ be the list of birthdays, with all $b \in L$.\\\\
\textsc{ALGORITHM}:\\
1. For $i=1,\dots,n$:
\tab If $Lookup(L[i])$ on $D$ returns true: \textsc{RETURN} $TRUE$.\\
\tab Else: $Insert(L[i])$ into $D$.\\
2. \textsc{RETURN} $FALSE$.\\

\pagebreak

%%%%% PROBLEM 2 %%%%%

\section{Problem 2}
Suppose you are writing a plagiarism detector. Students submit documents as part of a homework and each document is an (ordered) sequence of words. For some parameter $m$ decided by the provost, we say two documents are copies of each other if one of them uses a sequences of $m$ words (in that given order) from the other. Give an algorithm which given an integer $m$ and $N$ documents $D_1,\ldots,D_N$ as input, flags all submissions which are copies of some other submission. Your algorithm should run in expected $O(m + N + \text{total-length of documents})$ time (i.e., expected $O( m + N + n_1 + \ldots + n_N)$ time where $n_j$ is the length of $j$'th document) and be always correct.\\\\
INPUT:\\
$N = $ number of documents\\
$D = \{D_1,\dots,D_N\}$ documents\\
$m = $ length of sequence\\
Let $\mathcal{D}$ be the dictionary that stores $(key,value)$ pairs using hashing with chaining.\\
Let $key$ match to sequences of $m$ words.\\
Let $value$ match to integers.\\
Let $r = \sum_{i}n_i$.\\\\
\textsc{ALGORITHM}:\\
1. Initialize array $C$ of length $N$, where $C[i]=0 \ \forall i=\{1,\dots,N\}$.\\
2. For $i = 1,\dots,N$:\\
\tab For $j = 1,\dots,n_i-m$:\\
\tab \tab Let $key = (D_i[j,j+m])$.\\
\tab \tab Compute $value = Lookup(key)$.\\
\tab \tab If $value = NULL$: $Insert(key)$ into $\mathcal{D}$.\\
\tab \tab Else if $value\neq i$:, set $C[i]=1$ and $C[value]=1$.\\
3. \textsc{RETURN} $C$.\\

\pagebreak

%%%%% PROBLEM 3 %%%%%

\section{Problem 3}
Consider the problem \textsc{Find-Clique} defined as follows: ``Given a graph $G$ and a number $k$ as input, find a clique of size $k$ in $G$ if one exists.'' Recall the \textsc{Clique} decision problem from class: ``Given a graph $G$ and a number $k$, does $G$ contain a clique of size $k$?''. Give a polynomial-time reduction from \textsc{Find-Clique} to \textsc{Clique}.\\\\
We reduce \textsc{Find-Clique} to \textsc{Clique} by using a polynomial number of calls to a black-box that solves \textsc{Clique}.\\
Let \textsc{K-Clique} be the altered \textsc{Clique} algorithm such that the input graph $F$ into \textsc{K-Clique} is always size $k$; \textsc{K-Clique} only needs to check if $F$ is a clique.\\
Divide $G=(V,E)$ into $j$ subgraphs of size $k$. The maximum number of subgraphs of size $k$ is $n^k$.\\
Let $g_i \in \{g_1,\dots,g_j\}$ be the subgraphs of $G$ of size $k$.\\
\textsc{ALGORITHM}:\\
1. For $i=1,\dots,j$:\\
\tab Let $clique = $ \textsc{K-Clique}$(g_i)$.\\
\tab If $clique = TRUE$, \textsc{RETURN} $g_i$.\\
2. \textsc{RETURN} $FALSE$.\\\\
This algorithm reduces the problem into at most $n^k$ instances of \textsc{Clique}.\\

\pagebreak

%%%%% PROBLEM 4 %%%%%

\section{Problem 4}
Consider the problem \textsc{LPS} defined as follows: ``Given a matrix $A \in \mathbb{R}^{n \times n}$, a vector $b \in \mathbb{R}^n$ and an integer $k > 0$, does there exist a vector $x \in \mathbb{R}^n$ with at most $k$ non-zero entries such that $A \cdot x \geq b$''. Here $A \cdot x$ denotes the usual matrix-vector product and for two vectors $u,v$, we say $u \geq v$ if for every $i$, $u_i \geq v_i$. Give a polynomial-time reduction from \textsc{3SAT} to \textsc{LPS}.\\\\
We can show that \textsc{3SAT} $\leq_p$ \textsc{LPS} by showing that \textsc{Vertex-Cover} $\leq_p$ \textsc{LPS}.\\
Since \textsc{3SAT} $\leq_p$ \textsc{Vertex-Cover}, if \textsc{Vertex-Cover} $\leq_p$ \textsc{LPS}, then by transitivity \textsc{3SAT} $\leq_p$ \textsc{LPS}.\\
Let \textsc{M-LPS} be defined as: Given a matrix $A'\in R^{m \times n}$ and a vector $b'\in R^m$ and an integer $k > 0$, does there exist a vector $x \in R^n$ with at most $k$ non-zero entries such that $A'\cdot x \geq b'$.\\
Show that \textsc{vertex-cover} $\leq_p$ \textsc{M-LPS}:\\
1. Given an instance of vertex cover with graph $G$ and integer $k$, define an instance of \textsc{M-LPS} as: Let $G=(V,E)$ where $E=\{e_1,\dots,e_m\}$ are the edges in $G$ and $V=\{v_1,\dots,v_m\}$ are the vertices. Let $A'$ be the $m \times n$ matrix where $A'_{ij}=1$ if edge $e_i$ is adjacent to vertex $v_j$ and 0 otherwise. Let $b' \in R^n$ be the vector with all entries being 1.\\
\textsc{CLAIM}: $G$ has a vertex-cover of size $k$ iff there is a vector $y$ with at most $k$ non-zero entries such that $A'y\geq b$.\\
\textsc{PROOF}:\\
\tab \textsc{FORWARD}: Suppose there is a vertex cover $Y \subset [n]$ of size at most $k$. Let $y=1(Y)$ be the vector with $y_j=1$ if $j\in Y$ and 0 otherwise. For any $i \in [m]$, $(A'y)i=\sum_{j=1}^{n}(A'_{ij}y_j)=\sum_{j\in Y}A'_{ij} \geq 1$ (since $e_i$ should be adjacent to at least one vertex of $Y$).\\
\tab $\therefore A'y\geq b'$\\
\tab \textsc{BACKWARD}: Suppose there is a vector $y\in R^n$ with at most $k$ non-zero entries such that $A'y\geq b'$. Let $Y=\{j:y_j\neq0\}$. $|Y|\leq k$ and $Y$ is a vertex-cover. This is because for an index $i \in [m]$, $(A'y)_i\geq b_i = 1$ and $(A'y)_i\neq0$. We know that $(A'y)_i = \sum_{j=1}^{n}A'_{ij}y_j$, and for $(A'y)_i$ to be non-zero, one of the summands of $A'_{ij}y_j$ should be non-zero. This is only true when the edge $e_i$ is adjacent to a vertex in $Y$.\\
\tab $\therefore Y$ is a vertex-cover.\\
This shows that $G$ has a vertex-cover of size $k$ iff there is a vector $y$ with at most $k$ non-zero entries such that $A'y\geq b'$. Since an instance of \textsc{M-LPS} can be built in polynomial time, we can use a black-box for \textsc{M-LPS} to solve \textsc{vertex-cover}.\\
$\therefore$ \textsc{vertex-cover} $\leq_p$ \textsc{M-LPS}.\\
2. Consider an instance of \textsc{M-LPS} given by $A',b',k$. If $m=n$, we have an instance of $LPS$.\\
\tab If $m>n$, build a matrix $A\in R^{m \times m}$ by adding $m-n$ columns of length $m$ filled with zeroes to the matrix $A'$. A new instance of $LPS$ will be specified by $A,b',k$. We can check that the instance of \textsc{M-LPS} has a solution iff our constructed instance $A,b',k$ has a solution.\\
\tab If $m<n$, build a matrix $A \in R^{n\times n}$ by adding $n-m$ rows of length $n$ filled with zeroes to the matrix $A'$. Let $b \in R^n$ be the vector obtained by adding $n-m$ zeroes to the entries of $b'$. A new instance of \textsc{LPS} is specified by $A,b,k$. We can check that the instance of \textsc{M-LPS} has a solution iff our constructed instance $A,b',k$ of \textsc{LPS} has a solution.\\
We can use a black-box for \textsc{LPS} to solve \textsc{M-LPS} because the instances of \textsc{LPS} can be constructed in polynomial time.\\
\tab $\therefore$ \textsc{M-LPS} $\leq_p$ \textsc{LPS}.\\
Arguments 1 and 2 show that \textsc{vertex-cover} $\leq_p$ \textsc{LPS}.

\pagebreak

%%%%% PROBLEM 5 %%%%%

\section{Problem 5}
For this problem we need the notion of multi-variate polynomials over variables $x_1,\ldots,x_n$ and how they are specified. To review some terminology, we say a \emph{monomial} is a product of a real-number co-efficient $c$ and each variable $x_i$ raised to some non-negative integer power $a_i$: we can write this as $c x_1^{a_1} x_2^{a_2} \cdots x_n^{a_n}$. A polynomial is then a sum of a finite set of monomials.\\
We say a polynomial  $P$ is of degree at most $d$, if for any monomial $c x_1^{a_1} x_2^{a_2} \cdots x_n^{a_n}$ appearing in $P$, $a_1 + a_2 + \cdots a_n \leq d$. (For example, the degree of the previous polynomial is $7$). One can represent a $n$-variable polynomial of degree $d$ by at most $(n+1)^d$ numbers.\\
Consider the problem \textsc{Poly-Root} defined as follows: ``Given a polynomial with integer coefficients of degree at most $6$ as input, decide if there exists a $x \in \mathbb{R}^n$ such that $P(x) = 0$.'' Show that 3SAT reduces to \textsc{Poly-Root}. You don't have to write down the coefficients of the polynomials explicitly in your reduction - you can leave them as summations if it is more convenient for you.\\\\
For a term $y$, let $P_y$ be the polynomial defined as:\\
If $y=x_i$ for a variable: $P_y=1-x_i$.\\
Else if $y=\overline{x_i}$: $P_y=x_i$.\\
For every clause $C$, let $P_C$ be the polynomial obtained by multiplying the polynomials $P_y$ for every term $y$ that appears in $C$.\\
Given a \textsc{3SAT} instance $\phi = C_1 \vee \dots \vee C_k$, let $P_{\phi} = P_{C_1}^2 + \dots + P_{C_k}^2$.\\
\textsc{CLAIM}: $\phi$ is satisfiable iff there exists an $x \in R^n$ such that $P_{\phi}(x)=0$.\\
\textsc{PROOF}:\\
\tab \textsc{FORWARD}: Suppose that $\phi$ is satisfiable and let $a$ be a satisfying assignment for $\phi$. Then $P_{C_j}(a)=0$ for every clause $C_j$, since at least one of the terms in $C_j$ would be satisfied by $a$.\\
\tab $\therefore$ $P_{\phi}(a)=0$.\\
\tab \textsc{BACKWARD}: Suppose there exists a vector $b \in R^n$ such that $P_{\phi}(b)=0$. Define an assignment $a$ as:\\
\tab For each $i \in [n]$:\\
\tab \tab If $b_i \in \{0,1\}$: set $a_i=b_i$.\\
\tab \tab Else: set $a_i$ to 1.\\
\tab \textsc{CLAIM:} $a$ is a satisfying assignment for $\phi$.\\
\tab \textsc{PROOF:} If $P_{\phi}(b)=0$, then $P_{C_j}(b)=0$ for every $1 \leq j \leq k$. Let $C_j=y_{j1}\vee \dots \vee y_{j3}$. Since $P_{C_j} = P_{y_j1}P_{y_j2}P_{y_j3}$, $P_{C_j}(b)=0$ iff one of $P_{y_j1}(b),P_{y_j2}(b),P_{y_j3}(b)=0$. Any of these polynomials is zero iff the corresponding value of $b \in \{0,1\}$ and satisfies the associated term.\\
\tab $\therefore$ From our definition of $a$, $a$ satisfies the clause $C_j$. Since this applies to every clause, $a$ satisfies the CNF formula $\phi$.\\
The arguments show that $\phi$ has a satisfying assignment iff the polynomial $P_{\phi}$ has a root. Since we can build $P_{\phi}$ in polynomial time and it has degree at most 6, we can use a black-box for \textsc{POLY-ROOT} to solve \textsc{3SAT}.\\
$\therefore$ \textsc{3SAT} $\leq_p$ \textsc{POLY-ROOT}.\\

\end{document}