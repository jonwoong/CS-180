%%%%%%%%%% DOCUMENT STUFF %%%%%%%%%%

\documentclass[10pt,letterpaper]{article}
\usepackage{mathtools}
\usepackage{amsmath}
\usepackage{amssymb}
\usepackage{datetime}
\usepackage{setspace}
\usepackage{tkz-graph}
\usepackage[margin=1in]{geometry}

%%%%%%%%%% FORMATTING %%%%%%%%%%

\newdate{date}{22}{02}{2017}
\date{\displaydate{date}}
\doublespacing
\setcounter{secnumdepth}{0}
\newcommand\tab[1][0.5cm]{\hspace*{#1}}

%%%%%%%%%% CONTENT %%%%%%%%%%

%%%%% COVER PAGE %%%%%

\begin{document}
\title{CS 180: Homework 4}
\author{
	Jonathan Woong\\
	804205763\\
	Winter 2017\\
	Discussion 1B}
\maketitle
\pagebreak

%%%%% PROBLEM 1 %%%%%

\section{Problem 1}
When their respective sport is not in season, UCLA's student-athletes are very involved in their community, helping people and spreading goodwill for the school. Unfortunately, NCAA regulations limit each student-athlete to at most one community service project per quarter, so the athletic department is not always able to help every deserving charity. For the upcoming quarter, we have $S$ student-athletes who want to volunteer their time, and $B$ buses to help get them between campus and the location of their volunteering. There are $F$ projects under consideration; project $i$ requires $s_i$ student-athletes and $b_i$ buses to accomplish, and will generate $g_i > 0$ units of goodwill for the university.\\
Use dynamic programming to produce an algorithm to determine which projects the athletic department should undertake to maximize goodwill generated.  For full-credit, your algorithm should run in time $O(S B F)$ but you don't have to prove its correctness or analyze the time complexity.\\\\
$S$ student-athletes $=\{s_1, s_2, \dots, S\}$\\
$B$ buses $=\{b_1, b_2, \dots, B\}$\\
$F$ projects $=\{1, 2, \dots, F\}$\\
$G$ goodwill $=\{g_1, g_2, \dots, G\}$\\
\textit{Goal}: maximize $G$\\\\
Let $O$ be an optimal solution calculated using $OPT(project,student,bus)$:\\
\tab If project $n$ is not in $O$, we must solve the problem for projects $\{1,\dots,n-1\}$ using $OPT(n-1,s,b)$\\
\tab If project $n$ is in $O$, we must solve the problem for projects $\{1,\dots,n-1\}$, but we know that $OPT(n,s_n,b_n)=g_n+OPT(n-1,S-s_n,B-b_n)$\\\\
MAX-GOODWILL-COMPUTE-OPT:\\
1. Set $OPT(0,s,b)=0$ for all students $s=\{s_1,s_2,\dots,S\}$ and all buses $b=\{b_1,b_2,\dots,B\}$.\\
Set $OPT(i,0,0)=0$ for all $i=\{1,\dots,n\}$.\\
2. For $i=\{1,\dots,F\}$: $O(F)$\\
\tab For $s=\{s_1,\dots,S\}$: $O(S)$\\
\tab \tab For $b=\{b_1,\dots,B\}$: $O(B)$\\
\tab \tab \tab If $s_i > S$ or $b_i > B$: \\
\tab \tab \tab \tab $OPT(i,s,b)=OPT(i-1,s,b)$\\
\tab \tab \tab Else: \\
\tab \tab \tab \tab $OPT(i,s,b)=$max$
	\begin{cases}
		OPT(i-1,s,b) \\
		g_i+OPT(i-1,s-s_i,b-b_i)
	\end{cases}$\\
3. Output $OPT(F,S,B)$.\\\\
In order to find the subset that maximizes goodwill, we use the following algorithm:\\
FIND-SUBSET$(i,s,b)$:\\
1. If $s_i > s$ or $b_i > b$: \\
\tab RETURN FIND-SUBSET$(i-1,s,b)$.\\
2. Else: \\
\tab If $g_i + OPT(i-1,s-s_i,b-b_i) > OPT(i-1,s,b)$: \\
\tab \tab RETURN $\{i\} \cup $ FIND-SUBSET$(i-1,s-s_i,b-b_i)$.\\
\tab Else: \\
\tab \tab RETURN FIND-SUBSET$(i-1,s,b)$.\\ 

\pagebreak

%%%%% PROBLEM 2 %%%%%
\section{Problem 2}
You have a knapsack of total weight capacity $W$ and there are $n$ items with weights $w_1,\ldots,w_n$ respectively.  Give an algorithm to compute the number of different subsets that you can pack safely into the knapsack. In other words, given integers $w_1, \ldots, w_n, W$ as input, give an algorithm to compute the number of different subsets $S \subseteq [n]$ such that $\sum_{i \in S} w_i \leq W$.  For full-credit, your algorithm should run in time $O(nW)$ but you don't have to prove its correctness or analyze the time complexity.\\\\
Let $M$ be the set of items to be put in the knapsack with total weight $\leq W$.\\
Let $FEASIBLE(i,w)$ return the set of items in the knapsack with total weight $\leq W$. \\
If $w_n > W$:\\
\tab $FEASIBLE(n,w)=FEASIBLE(n-1,w)$\\
Else:\\
\tab If $n$ not in $M$:\\
\tab \tab $FEASIBLE(n,w)=FEASIBLE(n-1,w)$\\
\tab Else $n$ in $M$:\\
\tab \tab $FEASIBLE(n,w)=n \cup FEASIBLE(n-1,w-w_n)$\\\\
The algorithm to compute all possible subsets is therefore:\\
1. Set $FEASIBLE(0,w)=0$ for all weights $(w_1,\dots,w_n)$\\
2. Initialize stack $S$ to be empty.\\
3. For $i=1,\dots,n$:\\
\tab For $w=(w_1,\dots,w_n)$:\\
\tab \tab Compute $FEASIBLE(i,w)$ using the recurrence.\\
\tab \tab If $FEASIBLE(i,w)$ returns a set that is not currently in $S$:\\
\tab \tab \tab Push $FEASIBLE(i,w)$ onto $S$.\\
\tab \tab Else:\\
\tab \tab \tab Do nothing. \\
4. RETURN the total number of elements in $S$.\\\\
Just like the knapsack algorithm, this algorithm runs in $O(nW)$.

\pagebreak

%%%%% PROBLEM 3 %%%%%
\section{Problem 3}
Given two strings $X = x_1x_2\cdots x_m$ and $Y = y_1 y_2 \cdots y_n$, let $deleteScore(X,Y)$ be the least number of characters you have to delete from $X,Y$ so that you get two same strings. For example, if $X = goodman$ and $Y = goldmann$, then $deleteScore(X,Y) = 3$ (you can delete `o' from $X$, `l' and one `n' from $Y$ to get the same string - ``godman").\\
Give an algorithm that given two strings $X,Y$ computes $deleteScore(X,Y)$. For full-credit, your algorithm should run in polynomial time.\\\\
We define the recurrence relation as:\\
$deleteScore(X,Y)=$min$
	\begin{cases}
	case \ 1: 2+deleteScore(X-1,Y-1) \ if \ (X_m \neq Y_n) \\
	case \ 2: 0+deleteScore(X-1,Y-1) \ if \ (X_m = Y_n) \\
	case \ 3: 1+deleteScore(m-1,n) \\
	case \ 4: 1+deleteScore(m,n-1)
	\end{cases}$\\\\
This works much like the edit distance algorithm.\\
Case 1: Instead of incrementing the cost by 1 when $X_m \neq Y_n$, we increment it by 2. This is because the cost to create two matching strings requires the deletion of $X_m$ and $Y_n$.\\
Case 2: When $X_m$ and $Y_n$ match, we don't need to delete anything.\\
Case 3: When we insert a blank for $Y_n$ in the edit distance algorithm, this is equivalent to deleting $X_m$.\\
Case 4: When we insert a blank for $X_m$ in the edit distance algorithm, this is equivalent to deleting $Y_n$.\\

\pagebreak

%%%%% PROBLEM 4 %%%%%

\section{Problem 4}
There are four types of brackets: (,  ), $<$, and $>$. We define what it means for a string made up of these four characters to be {\em well-nested} in the following way:\\
(a) The empty string is well-nested.\\
(b) If A is well-nested, then so are $<$A$>$ and (A).\\
(c) If S, T are both well-nested, then so is their concatenation ST.\\
Devise an algorithm that takes as input a string $s = (s_1,s_2,...,s_n)$ of length $n$ made up of these four types of characters. The output should be the length of the shortest well-nested string that contains $s$ as a subsequence.\\\\
We use the same algorithm as RNA sequencing shown in class, but do not have the condition that a character can only be paired with another character more than 4 spaces away. Instead, we allow a closing bracket to be paired with a corresponding opening bracket at any prefix position.\\
Let $OPT(i,j)$ be the length of the shortest well-nested string that contains $s$.\\
$OPT(i,j)=$min$
\begin{cases}
case \ 1: 2 + OPT(i,j-1) \\
case \ 2: 1 + OPT(i,t-1)+OPT(t+1,j-1) \\
\end{cases}$\\
Case 1: $s_i$ is either a closing bracket that is not paired to an opening bracket or an opening bracket that was never paired with a closing bracket, so we +1 for the unpaired bracket and +1 for inserting the bracket that completes the pair.\\
Case 2: $s_i$ is a closing bracket that is paried to an opening bracket $s_t$, so we +1 for the closing bracket and calculate the optimal solution for the two substrings $(s_i,\dots,s_{t-1})$ and $(s_{t+1},\dots,s_{j-1})$.\\
We evaluate the subproblems in the order of small difference $(j-i)$ to bigger difference, just like the RNA sequencing algorithm:\\
ALGORITHM:\\
1. Initialize $OPT(i,k)=0$ for $i=0$ and $k=0$.\\
2. For $k=1,\dots,n-1$:\\
\tab For $i=1,\dots,n-k$:\\
\tab \tab $j = i + k$\\
\tab \tab Compute $OPT(i,j)$ using the recurrence.\\
3. RETURN $OPT(1,n)$.\\
Just like the RNA sequencing algorithm, this algorithm runs in $O(n^3)$.

\end{document}