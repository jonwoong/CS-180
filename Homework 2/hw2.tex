%%%%%%%%%% DOCUMENT STUFF %%%%%%%%%%

\documentclass[11pt,letterpaper]{article}
\usepackage{mathtools}
\usepackage{amsmath}
\usepackage{datetime}
\usepackage{setspace}
\usepackage[margin=1in]{geometry}

%%%%%%%%%% FORMATTING %%%%%%%%%%

\newdate{date}{01}{02}{2017}
\date{\displaydate{date}}
\doublespacing
\setcounter{secnumdepth}{0}
\newcommand\tab[1][1cm]{\hspace*{#1}}

%%%%%%%%%% CONTENT %%%%%%%%%%

%%%%% COVER PAGE %%%%%

\begin{document}
\title{CS 180: Homework 2}
\author{
	Jonathan Woong\\
	804205763\\
	Winter 2017\\
	Discussion 1B}
\maketitle
\pagebreak

%%%%% PROBLEM 1 %%%%%

\section{Problem 1}
(a) What is the FFT of (1,0,0,0)? What is the appropriate value of $w$ in this case? And of which sequence is (1,0,0,0) FFT? \\
$
FFT_4(1,0,0,0) \\
a = (1,0,0,0) \tab n = 4 \tab w = e^{2\pi i/n} = e^{2\pi i/4} = e^{\pi i/2} \\
a_0 = 1 \tab a_1 = 0 \tab a_2 = 0 \tab a_3 = 0 \\
\bar{s} = DFT_{n/2}(a_0, a_2) = DFT_{4/2}(1, 0) = DFT_2(1,0) = (1,1) \\
\tab a' = (1,0) \tab n=2 \tab w = e^{2\pi i/n} = e^{2\pi i/2} = e^{\pi i} \\
\tab a_0' = 1 \tab a_1' = 0 \\
\tab \bar{s} = DFT_{n/2}(a_0') = DFT_{2/2}(a_0') = DFT_1(a_0') = a_0' = 1 \\
\tab \bar{s}' = DFT_{n/2}(a_1') = DFT_{2/2}(a_1') = DFT_1(a_1') = a_1' = 0 \\$
\tab For $j=0$ to $[({n/2} - 1)=({2/2} - 1) = 0]: \\
\tab \tab r_j' = \bar{s}_j + w^j * \bar{s}_j' \\
\tab \tab \tab r_0' = \bar{s}_0 + w^0 * \bar{s}_0'  = 1 + (1 * 0) = 1 \\
\tab \tab r_{j+{n/2}}' = \bar{s}_j + w^{n/2} * w^j * \bar{s}_j' \\
\tab \tab \tab r_{0+{2/2}}' = r_1' = \bar{s}_0 + w^{1} * w^0 * \bar{s}_0' = 1 + (w*1*0) = 1 \\
\tab RETURN \ r' = (r_0', r_1') = (1,1) \\
\bar{s}' = DFT_{n/2}(a_1, a_3) = DFT_{4/2}(0,0) = DFT_2(0,0) = (0,0) \\
\tab a'' = (0,0) \tab n=2 \tab w = e^{2\pi i/n} = e^{2\pi i/2} = e^{\pi i} \\
\tab a_0'' = 0 \tab a_1'' = 0 \\
\tab \bar{s} = DFT_{n/2}(a_0'') = DFT_{2/2}(a_0'') = DFT_1(a_0'') = a_0' = 0 \\
\tab \bar{s}' = DFT_{n/2}(a_1'') = DFT_{2/2}(a_1'') = DFT_1(a_1'') = a_1' = 0 \\
$
\tab For $j=0$ to $[({n/2} - 1)=({2/2} - 1) = 0]: \\
\tab \tab r_j'' = \bar{s}_j + w^j * \bar{s}_j' \\
\tab \tab \tab r_0'' = \bar{s}_0 + w^0 * \bar{s}_0'  = 0 + (1 * 0) = 0 \\
\tab \tab r_{j+{n/2}}'' = \bar{s}_j + w^{n/2} * w^j * \bar{s}_j' \\
\tab \tab \tab r_{0+{2/2}}'' = r_1'' = \bar{s}_0 + w^{1} * w^0 * \bar{s}_0' = 0 + (w*1*0) = 0 \\
\tab RETURN \ r'' = (r_0'', r_1'') = (0,0) \\
$
For $j=0$ to $[({n/2} - 1)=(4/2 - 1) = (2 - 1) = 1]$: \\
\tab $r_j = \bar{s}_j + w^j * \bar{s}_j' \\
\tab \tab r_0 = \bar{s}_0 + w^0 * \bar{s}_0' = 1 + (1*0) = 1 \\
\tab \tab r_1 = \bar{s}_1 + w^1 * \bar{s}_1' = 1 + (w*0) = 1 \\
\tab r_{j+{n/2}} = \bar{s}_j + w^{n/2} * w^j * \bar{s}_j' \\
\tab \tab  r_{0+{4/2}} = r_2 = \bar{s}_0 + w^{2} * w^0 * \bar{s}_0' = 1 + (w^2 * 1 * 0) = 1 \\
\tab \tab  r_{1+{4/2}} = r_3 = \bar{s}_1 + w^{2} * w^1 * \bar{s}_1' = 1 + (w^2 * w * 0) = 1 \\
RETURN \ r = (1,1,1,1) \\ 
$ 
(b) Repeat for (1,0,1,-1). \\
$
FFT_4(1,0,1,-1) \\
a = (1,0,1,-1) \tab n = 4 \tab w = e^{2\pi i/n} = e^{2\pi i/4} = e^{\pi i/2} \\
a_0 = 1 \tab a_1 = 0 \tab a_2 = 0 \tab a_3 = 0 \\
\bar{s} = DFT_{n/2}(a_0, a_2) = DFT_{4/2}(1, 1) = DFT_2(1,1) =  \\
\tab a' = (1,1) \tab n=2 \tab w = e^{2\pi i/n} = e^{2\pi i/2} = e^{\pi i} \\
\tab a_0' = 1 \tab a_1' = 1 \\
\tab \bar{s} = DFT_{n/2}(a_0') = DFT_{2/2}(a_0') = DFT_1(a_0') = a_0' = 1 \\
\tab \bar{s}' = DFT_{n/2}(a_1') = DFT_{2/2}(a_1') = DFT_1(a_1') = a_1' = 1 \\$
\tab For $j=0$ to $[({n/2} - 1)=({2/2} - 1) = 0]: \\
\tab \tab r_j' = \bar{s}_j + w^j * \bar{s}_j' \\
\tab \tab \tab r_0' = \bar{s}_0 + w^0 * \bar{s}_0'  = 0 + (1 * 1) = 2 \\
\tab \tab r_{j+{n/2}}' = \bar{s}_j + w^{n/2} * w^j * \bar{s}_j' \\
\tab \tab \tab r_{0+{2/2}}' = r_1' = \bar{s}_0 + w^{1} * w^0 * \bar{s}_0' = 1 + (w*1*1) = 1+w \\
\tab RETURN \ r' = (r_0', r_1') = (2,1+w) \\
\bar{s}' = DFT_{n/2}(a_1, a_3) = DFT_{4/2}(0,-1) = DFT_2(0,-1) = (-1,-w) \\
\tab a'' = (0,-1) \tab n=2 \tab w = e^{2\pi i/n} = e^{2\pi i/2} = e^{\pi i} \\
\tab a_0'' = 0 \tab a_1'' = 0 \\
\tab \bar{s} = DFT_{n/2}(a_0'') = DFT_{2/2}(a_0'') = DFT_1(a_0'') = a_0' = 0 \\
\tab \bar{s}' = DFT_{n/2}(a_1'') = DFT_{2/2}(a_1'') = DFT_1(a_1'') = a_1' = -1 \\
$
\tab For $j=0$ to $[({n/2} - 1)=({2/2} - 1) = 0]: \\
\tab \tab r_j'' = \bar{s}_j + w^j * \bar{s}_j' \\
\tab \tab \tab r_0'' = \bar{s}_0 + w^0 * \bar{s}_0'  = 0 + (1 * -1) = -1 \\
\tab \tab r_{j+{n/2}}'' = \bar{s}_j + w^{n/2} * w^j * \bar{s}_j' \\
\tab \tab \tab r_{0+{2/2}}'' = r_1'' = \bar{s}_0 + w^{1} * w^0 * \bar{s}_0' = 0 + (w*1*-1) = -w \\
\tab RETURN \ r'' = (r_0'', r_1'') = (-1,-w) \\
$
For $j=0$ to $[({n/2} - 1)=(4/2 - 1) = (2 - 1) = 1]$: \\
\tab $r_j = \bar{s}_j + w^j * \bar{s}_j' \\
\tab \tab r_0 = \bar{s}_0 + w^0 * \bar{s}_0' = 2 + (1*-1) = -2 \\
\tab \tab r_1 = \bar{s}_1 + w^1 * \bar{s}_1' = (1+w) + (w*-w) = 1+w-w^2 = 2+i \\
\tab r_{j+{n/2}} = \bar{s}_j + w^{n/2} * w^j * \bar{s}_j' \\
\tab \tab  r_{0+{4/2}} = r_2 = \bar{s}_0 + w^{2} * w^0 * \bar{s}_0' = 2+(w^2*1*-1) = 2-w^2 = 3 \\
\tab \tab  r_{1+{4/2}} = r_3 = \bar{s}_1 + w^{2} * w^1 * \bar{s}_1' = (1+w)+(w^2*w*-w) = 1+w-w^4 = i \\
RETURN \ r = (-2,2+i,3,i) \\ 
$ 

\pagebreak

%%%%% PROBLEM 2 %%%%%
\section{Problem 2}
Run the BFS algorithm for the following graph with $s=1$ and $t=9: G=(V,E)$, where $V = \{1,2,3,4,5,6,7,8,9\}$, and $E = \{\{1,2\},\{1,3\},\{2,3\},\{2,4\},\{2,5\},\{3,5\},\{3,7\},\{4,5\},\{5,6\},\{8,9\}\}$. \\\\ 
\begin{center}
\begin{tabular} { |c|c|c|c|c|c|c|c|c|c| }
\hline
& 1 & 2 & 3 & 4 & 5 & 6 & 7 & 8 & 9 \\
\hline
$L_0$ = [s] & t & f & f & f & f & f & f & f & f \\
\hline
$L_1$ = \{2,3\} & t & t & t & f & f & f & f & f & f \\
\hline
$L_2$ = \{4,5\} & t & t & t & t & t & f & f & f & f \\
\hline
$L_3$ = \{7\} & t & t & t & t & t & f & t & f & f \\
\hline
$L_4$ = \{6\} & t & t & t & t & t & t & t & f & f \\
\hline
$L_5$ = $\emptyset$ & t & t & t & t & t & t & t & f & f \\
\hline
\end{tabular}
\end{center}

\pagebreak

%%%%% PROBLEM 3 %%%%%
\section{Problem 3}
Given a graph $G=(V,E)$ in adjacency list representation, give an algorithm that runs in time $O(|V|*|E|)$ to check if $G$ has a 'triangle', i.e., a triple of distinct vertics $\{u,v,w\}$ such that all three edges between them are present in $G$. \\\\
Input: adjacency list \\
Output: vertices ${u,v,w}$ of a triangle \\
Algorithm: For some edge given by $(u,v)$ in the adjacency list, explore vertex $w$'s list of neighbors ($w$ is any vertex that is not $u$ or $v$). If $w$ contains neighbors $u$ and $v$, then we have found a triangle. \\\\
For each edge $(u,v)$: $O(|E|)$ \\
\tab For each vertex $w$: $O(|V|)$ \\ 
\tab \tab if $w$ has neighbors $u$ and $v$: $O(1)$ \\
\tab \tab \tab return ${u,v,w}$ \\
return null \\
Total Runtime = $O(|E|)*O(|V|)*O(1)=O(|V|*|E|)$ \\

\pagebreak

%%%%% PROBLEM 4 %%%%%
\section{Problem 4}
Give an algorithm based on BFS that given a grpah $G=(V,E)$ (in adjacency list representation) checks whether or not $G$ has a cycle. Your algorithm should run in time $O(|V|+|E|)$. Prove your algorithm works. \\\\
Input: adjacency list \\
Output: true or false (contains cycle?) \\
Algorithm: essentially the same as BFS, but one change: \\
discovered[$u$] = false for all $u \neq s$ \\
discovered[$s$] = true \\
$L[0]=s\\ 
i\leftarrow0$ \\ 
While $L[i]$ is not empty: \\
\tab $L[i+1]\leftarrow \emptyset$ \\
\tab For each vertex $u \in L[i]$ \\
\tab \tab For each neighbor $v$ of $u (v \in A[u])$ \\
\tab \tab \tab if discovered[$v$] = true, then $CYCLE \ DETECTED$ \\
\tab \tab \tab else \\
\tab \tab \tab \tab set discovered[$v$]$\leftarrow$true\\
\tab \tab \tab \tab add $v$ to $L[i+1]$ \\
\tab \tab $i \leftarrow i+1$ \\
check if discovered[$t$] = true (not necessary for our application) \\
We use the BFS property that "all vertices with $DISCOVERED$ marked true when running BFS($s$) are the vertices in the connected component of $s$." This means that when exploring the vertices of $s$'s neighbor $u$ ($s$ connected to $u$ by definition), and $u$ discovers a neighbor $v$ ($u$ and $v$ connected by definition) that was previously marked $DISCOVERED$, that $s$ is also connected to $u$. This means that there is connectivity between $s, u, v$, forming a cycle. Since this algorithm takes no more time than the regular BFS, it will run in $O(|V|+|E|)$. \\ 

\end{document}